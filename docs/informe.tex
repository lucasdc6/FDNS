\documentclass[6pt]{article}

% Packages
\usepackage[utf8]{inputenc}
\usepackage{hyperref}
\usepackage{listings}
\usepackage[square,numbers]{natbib}
\usepackage[a4paper,margin=1in,footskip=0.25in]{geometry}

% Title
\title{Programación Funcional}
\author{Lucas Di Cunzolo}
\date{Febrero 2022}

% Document
\begin{document}

\maketitle

\section{Introducción}

En este trabajo se plantea la implementación parcial del protocolo DNS propuesto
por la RFC 1035 \cite{rfc1035}

\section{Motivación}

La motivación radica principalmente en querer aplicar los conceptos aprendidos en
la materia en un proyecto real. La elección del protocolo DNS viene de la intriga
de conocer un poco más en detalle la implementación del protocolo.

\section{Alcance}

El alcance de la implementación está acotado a los tipos de consultas más usados
entre los que se encuentran \textit{A}, \textit{AAAA}, \textit{MX}.

Además, el servidor DNS no aplica el uso de TTLs ni respuestas recursivas.

\section{Modelo de datos}

El proyecto se inició modelando un mensaje DNS tal cual se define en la RFC, en
el que se pueden distinguir 4 tipos bien definidos, el encabezado del mensaje

\begin{lstlisting}[
  frame=single,
  language=haskell,
  captionpos=b,
  caption=data DNSHeader - \textit{src/FDNS/Types.hs:81},
  label={lst:unchecked_free_movements}
]
data DNSHeader = DNSHeader {
  identifier          :: Word16,
  qr                  :: Bool,
  opcode              :: OPCODE,
  authoritativeAnswer :: Bool,
  truncatedMessage    :: Bool,
  recursionDesired    :: Bool,
  recursionAvailable  :: Bool,
  z                   :: Bool,
  rccode              :: RCODE,
  qdcount             :: Word16,
  ancount             :: Word16,
  nscount             :: Word16,
  arcount             :: Word16
} deriving (Show, Eq)
\end{lstlisting}

Dentro del encabezado, se pueden reconocer 2 tipos definidos que modelan el
tipo de operación (OPCODE) y el tipo de respuesta (RCODE).

El segundo tipo de dato definido es la pregunta

\clearpage

\begin{lstlisting}[
  frame=single,
  language=haskell,
  captionpos=b,
  caption=data DNSQuestion - \textit{src/FDNS/Types.hs:100},
  label={lst:unchecked_free_movements}
]
data DNSQuestion = DNSQuestion {
  qname               :: String,
  qtype               :: QTYPE,
  qclass              :: QCLASS
} deriving (Show, Eq)
\end{lstlisting}

En este caso, se cuenta con 2 tipos de datos que representan el tipo de
registro DNS (QTYPE) y su clase (QCLASS)

En tercer lugar, se modeló el concepto de recurso, que aplica tanto
para la sección de respuesta, de respuestas autoritativas y respuestas
adicionales.

\begin{lstlisting}[
  frame=single,
  language=haskell,
  captionpos=b,
  caption=data DNSQuestion - \textit{src/FDNS/Types.hs:106},
  label={lst:unchecked_free_movements}
]
data DNSResource = DNSResource {
  rname               :: String,
  rtype               :: QTYPE,
  rclass              :: QCLASS,
  ttl                 :: Word32,
  rdlength            :: Word16,
  rdata               :: String
} deriving (Show, Eq)
\end{lstlisting}

Por último, todos estos tipos de datos se combinan en el tipo de dato
que modela el mensaje DNS como tal

\begin{lstlisting}[
  frame=single,
  language=haskell,
  captionpos=b,
  caption=data DNSQuestion - \textit{src/FDNS/Types.hs:115},
  label={lst:unchecked_free_movements}
]
data DNSMessage = DNSMessage {
  header              :: DNSHeader,
  question            :: [DNSQuestion],
  answer              :: [DNSResource],
  authority           :: [DNSResource],
  additional          :: [DNSResource]
} deriving (Show, Eq)
\end{lstlisting}

\section{Conceptos de la materia}

Para el desarrollo de este proyecto, se utilizaron conceptos aprendidos
en la materia tales como

\subsection{Recursión estructural}

Utilizada para el desarrollo de las funciones de desempaquetado, ya que
es necesaria una forma de recorrer la trama de bytes, acotando en
cada iteración el conjunto de bytes. En este caso, la condición de
corte está dada por la información provista por el encabezado.

Luego, se utiliza indirectamente en varias de las funciones que
hacen uso de funciones tales como \textit{map} o \textit{foldl}

\subsection{Funciones de alto orden}

Aplicadas ampliamente a lo largo del proyecto, me permitió desacoplar
mucho más la lógica en varios puntos, para desarrollar funciones
genéricas que permitan aplicarse en varios casos. Uno de los casos
más claros son las funciones \textit{packList} y \textit{unpackList}
usadas para trabajar con todos los recursos representados como listas
del mensaje DNS (preguntas, respuestas, respuestas autoritativas
y respuestas adicionales)

\subsection{Funciones currificadas}

Como sabemos, en haskell todas las funciones son currificadas, esto nos
permite escribir código compacto y potente. A lo largo de trabajo se
intentó desarrollar un código reutilizable.

\subsection{Pattern matching}

Utilizado ampliamente en la definición de las funciones presentadas
como utilidades para interactuar con los tipos algebraicos definidos.

\section{Bibliotecas y módulos usados}

A continuación se van a describir las principales bibliotecas y módulos que se
utilizaron en el desarrollo de este proyecto.

\subsection{network \cite{package-network-socket}}

Para implementar el servidor UDP se utilizaron los módulos \textit{Network.Socket}
y \textit{Network.Socket.ByteString} pertenecientes al paquete \textit{network}

Esta biblioteca se desarrollo una función de top-level que interactúa con
los datos recibidos por el cliente a través de un socket UDP.

\subsection{GetOpt \cite{package-get-opt}}

La interfaz definida para este programa es a través de líneas
de comandos (CLI), para desarrollarla, se utilizó el módulo
\textit{System.Console.GetOpt}, perteneciente al paquete
\textit{base} de haskell.

Para esto, se definió un tipo \textit{Option} el cual representa todas
las opciones posibles.

\begin{lstlisting}[
  frame=single,
  language=haskell,
  captionpos=b,
  caption=data Options - \textit{src/FDNS/Commands.hs:14},
  label={lst:unchecked_free_movements}
]
data Options = Options {
  optConfig       :: FilePath,
  optBindAddress  :: String,
  optPort         :: String,
  optHelp         :: Bool
} deriving Show
\end{lstlisting}

A su vez, se definió una función que declara la descripción de las
opciones (\textit{OptDescr}). Esta biblioteca permitió armar una interfaz
completa, que ya contempla el parseo de los argumentos, simplemente
pasandole los argumentos que se recuperan directamente usando la función
\textit{getArgs}.

\begin{lstlisting}[
  frame=single,
  language=haskell,
  captionpos=b,
  caption=Ejemplo de uso,
  label={lst:unchecked_free_movements}
]
data Options = Options {
  optConfig       :: FilePath,
  optBindAddress  :: String,
  optPort         :: String,
  optHelp         :: Bool
} deriving Show
\end{lstlisting}

\subsection{Yaml \cite{package-yaml}}

El modo de configuración de los registros DNS del servidor se quiso
mantener simple, por lo que se definió un archivo de configuración
básico en formato YAML, descrita en la siguiente sección.

\subsubsection{Definición}

\clearpage
\begin{lstlisting}[
  frame=single,
  captionpos=b,
  caption=Ejemplo de configuración,
  label={lst:unchecked_free_movements}
]
# Listado de dominios
domains:
  # Dominio en cuestion
  - name: .example.com
    # Listado de registros del dominio
    records:
        # Tipo del dominio
        # Puede ser A, AAAA, MX o TXT
      - type: A
        # Valor del registro
        # El formato depende del tipo de registro
        value: 1.2.3.4
\end{lstlisting}

Teniendo esto definido, ahora podemos llamar a nuestro
servidor de la siguiente forma

\begin{lstlisting}[
  frame=single,
  captionpos=b,
  caption=Ejemplo de llamado usando la CLI y un archivo YAML,
  label={lst:unchecked_free_movements}
]
./fdns --config ./example-config.yaml
\end{lstlisting}

\subsubsection{Desarrollo}

Para poder parsear los archivos yaml, se crearon los tipos de datos
que representan cada uno de los objetos del yaml planteados

\begin{lstlisting}[
  frame=single,
  language=haskell,
  captionpos=b,
  caption=Tipos de datos YAML,
  label={lst:unchecked_free_movements}
]
data Record = Record {
  recordType    :: String,
  value         :: String
} deriving (Eq, Show)

data Domain = Domain {
  name          :: String,
  records       :: [Record]
} deriving (Eq, Show)

data Config = Config {
  domains :: [Domain]
} deriving (Eq, Show)
\end{lstlisting}

Junto a eso, a cada uno de estos nuevos tipos, se le definieron
las funciones \textit{parseJSON} de la clase \textit{FromJSON}

\clearpage
\begin{lstlisting}[
  frame=single,
  language=haskell,
  captionpos=b,
  caption=Funciones parseJSON,
  label={lst:unchecked_free_movements}
]
instance FromJSON Record where
  parseJSON (Y.Object v) =
    Record <$>
    v .:   "type"       <*>
    v .:   "value"
  parseJSON k = fail ("Expected Object for Config value: " ++ show k)

instance FromJSON Domain where
  parseJSON (Y.Object v) =
    Domain <$>
    v .:   "name"       <*>
    v .:   "records"
  parseJSON k = fail ("Expected Object for Config value: " ++ show k)

instance FromJSON Config where
  parseJSON (Y.Object v) =
    Config <$>
    v .:   "domains"
  parseJSON _ = fail "Expected Object for Config value"
\end{lstlisting}

Por último, se desarrolló una función que, usando la función
\textit{decodeFileEither}, se encarga de leer el archivo de configuración
en formato yaml, y retornar, un \textit{Config} usando la mónada IO

\begin{lstlisting}[
  frame=single,
  language=haskell,
  captionpos=b,
  caption=Funcion readConfig,
  label={lst:unchecked_free_movements}
]
readConfig :: String -> IO Config
readConfig configFilePath =
  either (error . show) id <$>
  (decodeFileEither configFilePath)
\end{lstlisting}

\subsection{co-log \cite{package-co-log}}

Para mejorar la visibilidad del funcionamiento del servidor, se
plantearon una serie de logs, usando la biblioteca co-log. La
ingtegración fue bastante simple usando el logger ya provisto
por el módulo \textit{Colog}

\begin{lstlisting}[
  frame=single,
  language=haskell,
  captionpos=b,
  caption=Ejemplo de uso de co-log,
  label={lst:unchecked_free_movements}
]
let logger = logStringStdout
logger <& ("Load config file from \"" ++ configFile ++ "\"")
\end{lstlisting}

\subsection{Hspec \cite{package-hspec}}

Para facilitar el desarrollo de nuevas funcionalidades, se estableció
un flujo de TDD, en donde toda nueva funcionalidad que se quiera
introducir, primero tiene que ser planteada en formato de test.

En este caso, se aplicó test sobre las funciones principales de
empaquetado y desempaquetado, ya que es simple plantear una
correspondencia entre un conjunto de bytes en formato ByteString, con
el tipo \textit{DNSMessage}.

Para simplificar la configuración de la librería, se utilizó una
estructura de directorios estándar, en donde, cada módulo de nuestro
código se corresponde con un módulo de spec. El mapeo se puede encontrar
en el archivo de inicio de los tests \textit{tests/Spec.hs}

\begin{lstlisting}[
  frame=single,
  language=haskell,
  captionpos=b,
  caption=Archivo tests/Spec.hs,
  label={lst:unchecked_free_movements}
]
{-# OPTIONS_GHC -F -pgmF hspec-discover #-}
\end{lstlisting}

Además, se agregó el apartado \textit{test-suite} en cabal.

\begin{lstlisting}[
  frame=single,
  language=haskell,
  captionpos=b,
  caption=Apartado test-suite del archivo fdns.cabal,
  label={lst:unchecked_free_movements}
]
test-suite spec
  type:                 exitcode-stdio-1.0

  -- .hs or .lhs file containing the Main module.
  main-is:              Spec.hs

  other-modules:      FDNS.Parsers.PackSpec
                    , FDNS.Parsers.UnpackSpec
                    , FDNS.UtilsSpec

  -- Directories containing source files.
  hs-source-dirs:       tests

  -- Other library packages from which modules are imported.
  build-depends:      base >=4.10.1.0 && <4.15
                    , bytestring
                    , fdns
                    , hspec

  -- Base language which the package is written in.
  default-language:     Haskell2010

  build-tool-depends: hspec-discover:hspec-discover == 2.*
\end{lstlisting}

Esto nos permite ejecutar el conjunto de tests de la siguiente forma

\begin{lstlisting}[
  frame=single,
  captionpos=b,
  caption=Ejemplo de ayuda,
  label={lst:unchecked_free_movements}
]
cabal test all
\end{lstlisting}


\section{Interfaz}

La interfaz se planteó de forma que sea simple empezar a probar el
servidor DNS, contando con un archivo de configuración como el
planteado en el apartado de YAML.

\begin{lstlisting}[
  frame=single,
  captionpos=b,
  caption=Ejemplo de ayuda,
  label={lst:unchecked_free_movements}
]
$ fdns --help
Usage: fdns -c FILE [-H HOST] [-p PORT]
  -h, -?   --help         Show this help message
  -c FILE  --config=FILE  Config file
  -H HOST  --host=HOST    Address to bind
  -p PORT  --port=PORT    Port to bind
\end{lstlisting}

\section{Desarrollo del proyecto}

El proyecto se planteó separando completamente la lógica del protocolo
DNS, de la interacción con el cliente (socket UDP) y el usuario (CLI).

Para esto, se cuenta con una función \textit{main} como punto de entrada,
en donde se parsean los argumentos usando \textit{GetOpt}, y dependiendo
los argumentos enviados, se muestra una ayuda, o se inicia el servidor
UDP.

El servidor UDP toma todas sus configuraciones necesarias de los
argumentos al programa, e inicia un socket a la espera de datos.

Una vez recuperados los datos, en formato ByteString, se llama a la
función de desempaquetado.

\section{Desarrollo del protocolo}

Las funciones referentes al protocolo se pueden diferenciar en 2 grandes
gruopos, las de desempaquetado, que toman una trama de bytes, y retornan
uno de los tipos definidos del modelo de datos. En su contraparte, tenemos
las funciones de empaquetado, que reciben tipos del modelo de datos, y
retornan su representación en bytes.

\subsection{Desempaquetado}

El primer paso es identificar y desempaquetar los primeros
12 bytes, que representan el encabezado del mensaje. En el mismo vamos
a tener toda la información necesaria para poder trabajar con el resto
del mensaje.

El encabezado tiene el siguiente formato

\begin{lstlisting}[
  frame=single,
  captionpos=b,
  caption=Ejemplo de ayuda,
  label={lst:unchecked_free_movements}
]
            1  1  1  1  1  1
            0  1  2  3  4  5  6  7  8  9  0  1  2  3  4  5
            +--+--+--+--+--+--+--+--+--+--+--+--+--+--+--+--+
            |                      ID                       |
            +--+--+--+--+--+--+--+--+--+--+--+--+--+--+--+--+
            |QR|   Opcode  |AA|TC|RD|RA|   Z    |   RCODE   |
            +--+--+--+--+--+--+--+--+--+--+--+--+--+--+--+--+
            |                    QDCOUNT                    |
            +--+--+--+--+--+--+--+--+--+--+--+--+--+--+--+--+
            |                    ANCOUNT                    |
            +--+--+--+--+--+--+--+--+--+--+--+--+--+--+--+--+
            |                    NSCOUNT                    |
            +--+--+--+--+--+--+--+--+--+--+--+--+--+--+--+--+
            |                    ARCOUNT                    |
            +--+--+--+--+--+--+--+--+--+--+--+--+--+--+--+--+
\end{lstlisting}

Esto se traduce de la siguiente forma
\begin{itemize}
  \item Los primeros 2 bytes se usan para representar el ID
  \item El tercer byte se usa para representar 4 campos
  \begin{itemize}
    \item El primer bit representa si es una query (0) o una respeusta (1)
    \item Los siguientes 4 bits, representan el código de operación
    \item El sexto bit se utiliza solamente en las respeustas,
    y represnta si es una respeusta autoritativa.
    \item El septimo bit se utiliza para marcar que el mensaje va
    a estar truncado por el tamaño máximo del canal.
  \end{itemize}
  \item El cuarto byte se usa para representar otros 4 campos
  \begin{itemize}
    \item El primer bit marca si se quiere realizar una pregunta
    recursvia.
    \item El segundo bit, es modificado por la respuesta, y marca si
    el servidor sopoerta respuestas recursivas.
    \item Los siguientes 3 bits, estan reservados para futuras
    implementaciones
    \item Los siguientes 4 bits representan el tipo de respuesta
  \end{itemize}
  \item El quinto y sexto byte representan la cantidad de preguntas
  que contiene el mensaje.
  \item El septimo y octavo byte representan la cantidad de respuestas
  que contiene el mensaje.
  \item El noveno y décimo byte representan la cantidad de respuestas
  autoritativas que contiene el mensaje.
  \item El onceavo y doceavo byte representan la cantidad de respuestas
  adicionales que contiene el mensaje.
\end{itemize}

Esto se traduce en una función destinada a desempaquetar el encabezado,
que recibe los bytes en el formato \textit{ByteString}, y retorna un
\textit{DNSHeader}

En este caso, separamos cada uno de los bytes usando una función
definida en el módulo \\\textit{FDNS.Parsers.Internal.Utils}, la
cual añade una capa de control usando la mónada \textit{Maybe}.

Una vez separados los bytes, se empieza a construir el encabezado
recuperando los valores con \textit{fromMaybe}, tomando por defecto
el valor nulo de los campos. Esto permite armar siempre un encabezado
bien formado para retornar, aunque la trama de bytes sea incorrecta,
marcando el error en el campo \textit{rcode} del ebncabezado.

El siguiente paso, y haciendo uso de los datos recuperados
anteriormente, es desempaquetar tantas preguntas como el
valor de \textit{qdcount} del encabezado. Para esto, se
definió una función de alto orden que hace uso de
recursión estructural para recorrer la trama de bytes e
ir desempaquetando aplicando la función que recibe como
primer parámetro

\begin{lstlisting}[
  frame=single,
  language=haskell,
  captionpos=b,
  caption=Funcion unpackList y unpackQuestions,
  label={lst:unchecked_free_movements}
]
unpackList :: (BS.ByteString -> (Either DNSError a, BS.ByteString))
              -> Word16
              -> BS.ByteString
              -> [Either DNSError a]
unpackList unpack 0 bytes = []
unpackList unpack n bytes = let (resource, bytes')  = unpack bytes in
                            resource : unpackList unpack (n-1) bytes'

unpackQuestions :: Word16 -> BS.ByteString -> [Either DNSError DNSQuestion]
unpackQuestions = unpackList unpackQuestion
\end{lstlisting}

En este caso, vamos a recorrer la trama de bytes en cada llamado
de la recursión, usando la función \textit{unpackQuestion}, sobre una
subtrama de bytes diferente, y con la condición de corte
dada por que la cantidad de preguntas a desempaquetar llegue a 0.

Esta función \textit{unpackQuestion} se aplica para desempaquetar
las preguntas, que tienen el siguiente formato

\begin{lstlisting}[
  frame=single,
  language=haskell,
  captionpos=b,
  caption=Formato preguntas DNS,
  label={lst:unchecked_free_movements}
]
            1  1  1  1  1  1
            0  1  2  3  4  5  6  7  8  9  0  1  2  3  4  5
            +--+--+--+--+--+--+--+--+--+--+--+--+--+--+--+--+
            |                                               |
            /                     QNAME                     /
            /                                               /
            +--+--+--+--+--+--+--+--+--+--+--+--+--+--+--+--+
            |                     QTYPE                     |
            +--+--+--+--+--+--+--+--+--+--+--+--+--+--+--+--+
            |                     QCLASS                    |
            +--+--+--+--+--+--+--+--+--+--+--+--+--+--+--+--+
\end{lstlisting}

Por último, y de forma similar al proceso anterior, se tienen que
desempaquetar las respuestas, respuestas autoritativas y respuestas
adicionales que, recordando la definición del protocolo, comparten
su definición.

\begin{lstlisting}[
  frame=single,
  language=haskell,
  captionpos=b,
  caption=Función unpackResources,
  label={lst:unchecked_free_movements}
]

unpackResources :: Word16 -> BS.ByteString -> [Either DNSError DNSResource]
unpackResources = unpackList unpackResource
\end{lstlisting}

En este caso, usamos la función \textit{unpackResource},
aplicada con una trama de bytes que tiene que repretar
el siguiente formato

\clearpage
\begin{lstlisting}[
  frame=single,
  language=haskell,
  captionpos=b,
  caption=Formato preguntas DNS,
  label={lst:unchecked_free_movements}
]
            1  1  1  1  1  1
            0  1  2  3  4  5  6  7  8  9  0  1  2  3  4  5
            +--+--+--+--+--+--+--+--+--+--+--+--+--+--+--+--+
            |                                               |
            /                                               /
            /                      NAME                     /
            |                                               |
            +--+--+--+--+--+--+--+--+--+--+--+--+--+--+--+--+
            |                      TYPE                     |
            +--+--+--+--+--+--+--+--+--+--+--+--+--+--+--+--+
            |                     CLASS                     |
            +--+--+--+--+--+--+--+--+--+--+--+--+--+--+--+--+
            |                      TTL                      |
            |                                               |
            +--+--+--+--+--+--+--+--+--+--+--+--+--+--+--+--+
            |                   RDLENGTH                    |
            +--+--+--+--+--+--+--+--+--+--+--+--+--+--+--+--|
            /                     RDATA                     /
            /                                               /
            +--+--+--+--+--+--+--+--+--+--+--+--+--+--+--+--+
\end{lstlisting}

\subsection{Trabajando con el tipo DNSMessage}

Una vez analizado el mensaje DNS, se tiene que recuperar, de existir,
el registro que satisfaga la pregunta. Esto lo vamos a hacer usando
la función \textit{lookup}

\begin{lstlisting}[
  frame=single,
  language=haskell,
  captionpos=b,
  caption=Ejemplo de ayuda,
  label={lst:unchecked_free_movements}
]
lookup :: Config -> String -> String -> [Record]
lookup config name' recordType' =
  case find (\d -> name d == name') (domains config) of
    (Just domain) -> filter
                      (\r -> recordType r == recordType')
                      (records domain)
    Nothing       -> []
\end{lstlisting}

Con esto vamos a tener una lista de \textit{Record}, que
es fácilmente transformable a una lista de \textit{DNSResource}
usando un map y una función auxiliar que dado un \textit{Record}
retorna un \textit{DNSResource}

Por último, para agregarlo, se tiene un conjunto de funciones
que reciben un \textit{DNSMessage} y retornan un nuevo
\textit{DNSMessage} modificado con ese agregado

\begin{lstlisting}[
  frame=single,
  language=haskell,
  captionpos=b,
  caption=Ejemplo de ayuda,
  label={lst:unchecked_free_movements}
]
-- Agregar una lista de preguntas al mensaje
(<<?) :: DNSMessage -> [DNSQuestion] -> DNSMessage
-- Agregar una lista de respuestas al mensaje
(<<!) :: DNSMessage -> [DNSResource] -> DNSMessage
-- Agregar una lista de respuestas autoritativas al mensaje
(<<@) :: DNSMessage -> [DNSResource] -> DNSMessage
-- Agregar una lista de preguntas adicionales al mensaje
(<<+) :: DNSMessage -> [DNSResource] -> DNSMessage
\end{lstlisting}

Para simplificar la construcción del nuevo mensaje DNS, se cuenta con
la función \textit{dnsResolver}, la cual ya se encarga de recuperar
los registros para las preguntas dadas, y retorna un nuevo mensaje
DNS con las respuetas ya agregadas.

\clearpage
\begin{lstlisting}[
  frame=single,
  language=haskell,
  captionpos=b,
  caption=Ejemplo de ayuda,
  label={lst:unchecked_free_movements}
]
dnsResolver :: Config -> DNSMessage -> DNSMessage
dnsResolver config message = message <<! resources
  where questions = question message
        lookupConfig = FDNS.Config.lookup config
        resources = concat $ lookupConfig <$> questions
\end{lstlisting}

\subsection{Empaquetado}

Para poder retornar la respuesta al cliente, es necesario empaquetar
el mensaje en una trama de bytes. Para ello, tenemos funciones similares
a las vistas en el desempaquetado.

Contamos con una función \textit{packHeader} que recibe un \textit{DNSQuestion},
y retorna un ByteString.

Luego contamos con una función genérica para empaquetar listas, la cual es usada
para empaquetar tanto las preguntas (\textit{packQuestions}), como para empaquetar
el resto de los recursos (\textit{packResoureces})

\begin{lstlisting}[
  frame=single,
  language=haskell,
  captionpos=b,
  caption=Funcion unpackList y unpackQuestions,
  label={lst:unchecked_free_movements}
]
packList :: (a -> BS.ByteString) -> [a] -> BS.ByteString
packList pack = foldl
                (\bytes resource -> BS.append bytes (pack resource))
                BS.empty

packQuestions :: [DNSQuestion] -> BS.ByteString
packQuestions = packList packQuestion

packResoureces :: [DNSResource] -> BS.ByteString
packResoureces = packList packResourece
\end{lstlisting}

\section{Conclusiones}

El planteamiento del modelo de datos fue muy simple a nivel conceptual,
y el lenguaje ayudó en gran medida a bajar rapidamente varias ideas
para el desarrollo. Realmente me ubiera gustado contar con más tiempo
para poder llegar a plantear un menejo de errores más completo en cuanto
al protocolo DNS, como así para plantear una mirada más completa del
mismo, aunque creo que sirvió como una primera introducción para empezar
a entender como pueden plantearse proyectos con una complejidad media,
teniendo una separación fuerte de la lógica funcional, de la interacción
con el mundo exterior.

\subsection{Para mejorar}

Como se nombró anteriormente, un manejo de los errores más completo.
También creo que la lógica de desempaquetado de un mensaje se puede
mejorar para llevarlo a un plano más simple, sobre todo en el manejo
de los corrmientos entre secciones del mensaje.

Otro detalle a tener en cuenta sería introducir alguna estructura de
datos que permita trabajar de forma simple, e implementar el concepto
de compresión de los mensajes, actualmente, los mensajes generados
por este servidor DNS van a ser relativamente más pesados ya que en
cada aparición de un dominio, se transforma directamente a su
representación en bytes. Se había hablado de introducir los
Zipper \cite{package-zipper}.

\clearpage
\renewcommand\refname{Referencias}
\bibliographystyle{unsrtnat}
\bibliography{refs}

\end{document}
